\subsection{Overview}
In accordance with the Health and Safety Executive guidance, EMF will
provide, at any one time, 4 first-aiders, available and on call 24 hours a day
for the duration of the festival. Their remit is threefold:

\begin{itemize}
  \item Provision of first aid to any festival attendees, and EMF staff or
  volunteers, throughout the duration of the festival and during setup and
  strike-down
  \item Support for the basic welfare of the festival attendees and EMF
  staff and volunteers
  \item Management of any situations involving lost children for the duration of
  the event, in conjunction with the Children team. (see Appendix \ref{lost-child-policy})
\end{itemize}

First Aid practice is defined by the 10th Edition of the First Aid Manual
(Published 2014, Dorling Kindersley), the official manual of the Red Cross, St
John’s Ambulance and St Andrew’s Ambulance. First-aid volunteers will be
required to be familiar with this document and operate within its instructions.
A copy of the manual will be available on site for reference.


\subsection{Recruitment}
All first-aiders at EMF are volunteers and must present two different kinds of
credentials to the First Aid Team Lead: Each volunteer must have, as a minimum,
a qualification that meets the guidelines and criteria as defined by the Health
\& Safety Executive (HSE) in respect of the 1981 (First Aid) Regulations, such
as a First Aid at Work certificate.

Those with qualifications that are equivalent to, or superior than,
first-aid-at-work will also be accepted. Examples include:

\begin{itemize}
  \item Healthcare professionals, for example GPs, nurses, or paramedics
  \item Community First Responder
  \item First aider with St John’s Ambulance
  \item First aider with the Red Cross
\end{itemize}

In all cases, a copy of the relevant qualification(s) will be checked by the
First Aid Team Lead before the event. A digital copy will be retained by the
First Aid Team Lead and the event organiser in a secure format in accordance
with the Data Protection Act.

\subsection{Disclosure}
As first aid volunteers may need to work with children or vulnerable adults an
Enhanced Disclosure from the Disclosure and Barring Service is required. Proof
of this must be presented to the First Aid Team Lead and the festival organiser
before the event. These without a current Enhanced Disclosure will be required
to apply for one through the EMF  before the event. Digital copies of disclosure
paperwork for all volunteers will be kept securely by the First Aid Team Lead and
EMF organiser in accordance with the Data Protection Act.

\subsection{Strategy}
First aid cover will be provided 24 hours a day for the days when the festival
is open to the public. In addition, cover will be provided for the setup and
strike-down of the festival as long as EMF staff and volunteers are on site.

Cover will be organised as 3 shifts of 8 hours, with four first aiders on duty
at any one time. First aiders will operate in pairs, with one pair roaming the
site, and the other pair based at the designated first aid point. Both teams
will be provided with radios.

The shifts will commence at 10am in the morning. The 2am to 10am shift will
be covered by two first aiders and will be an `on-call’ service, i.e a mobile
phone number will be provided to all the EMF personnel and stewards for the
first aiders on duty.

One member of the four will be designated the ``shift-leader'', responsible for
the organisation of the two teams for the duration of the shift.

The first aid point will also operate as a welfare point. It will therefore be
equipped with bottled water, suncream, camp beds and heaters (see below).

The first aid point will be in a dedicated tent, clearly marked on the site
plan, and signposted outside. Its location will be made known to EMF staff,
volunteers and attendees and will be staffed from 10am to 2am through-out
the event, with the overnight first aid cover provided on an on-call basis.

\subsection{Access}

Access to the site is via road. The first aid point will be located near the
entrance to the site. In the event of an ambulance being required on scene, the
first-aid team will liase with the stewards to ensure the area of road around
the site entrance is cleared.

Non-urgent casualties who require hospital treatment but not an ambulance will
be taken to the receiving hospital via car, either by a car belonging to the
first-aid team, or a taxi. In both cases, the casualty will be escorted by a
member of the first-aid team, if requested to do so by the casualty or if the
first-aider deems it necessary.

The site itself is a mixture of fields and tracks. There are
trackways throughout the site, with extra firelanes marked for emergency
vehicles. In the event an ambulance is required at another part of the site,
the first-aid team will liase with the stewards to direct the vehicle and
provide clear access.

\subsection{Communications}

The First-aid team will work closely with the volunteer, info-desk and children
teams.

Volunteers and stewards will be made aware of the location of the first-aid
point and the mobile phone numbers for the first aid team. Both of the first
aid pairs on duty will be provided with a mobile phone and these two numbers
will be circulated to the info-desk, volunteer and children teams.

Volunteers and members of the general EMF team who respond to a first-aid
situation should advise the casualty to make their way to the first aid point
if they are able to do so. If not, the volunteer should contact the first aid
team immediately and stay with the casualty until members of the first aid team
arrive.

Between the hours of 2am and 10am, the first-aid team will be available via the
first aid mobile number. Volunteers responding to first aid situations at this
point should proceed as above, but call the first aid mobile number
immediately.

Both the first aid point and the roaming team will be provided with a radio.
Radio protocols, such as channels, codewords and such will be circulated at the
initial briefing, on-site. Particular attention is given to the lost and found
child reporting (see Lost Children Policy).

\subsection{Equipment}
Equipment will be provided for each first aid volunteer. These will include a
first aid kit, a high-visibility tabard and a radio. Volunteers will be advised
to bring appropriate warm and waterproof clothing and footwear.

First aid supplies and equipment will be purchased from St Johns Ambulance
supplies. The First Aid Team Lead will be responsible for the first aid
equipment and consumables for the team, and checking that it meets the required
standards.

\subsection{Medical waste disposal}
All medical waste will be disposed of in the correctly marked bags (orange,
clinical waste bags) and will be kept at the first aid point until the end of
the strike-down, whereupon they will be disposed of in accordance with the
local health authority's requirements. Sharps bins will be provided and
disposed of in the same fashion if used.

Standard waste will be disposed of using the facilities provided by the saniation team.

\subsection{Reporting}
All first aid administered will be recorded electronically on the
‘patient-report-form', with any serious incidents being reported on an
additional RIDDOR form. Patient report forms will be kept by the First Aid Team
Lead and the EMF organiser with a copy sent to the patient upon request. This
data will be kept in a secure format and accordance with the Data Protection
Act.

These members of the first aid team who are members of St Johns, Red Cross or another
organisaton who have their own paperwork requirements may fill out their own paperwork
in addition to EMF's requirements.

\subsection{Local Authorities}
Both the local ambulance service control room and the local police service will
be informed, prior to festival, that an event is taking place where first aid
cover is being provided.

\subsubsection{Hospitals}

The nearest hospital is:

Royal Surrey County Hospital, Egerton Rd, Guildford, Surrey GU2 7XX

Royal Surrey County Hospital is located 5 miles away from the EMF site.

First Aid volunteers will be made aware of the hospital location. In addition,
the local ambulance service shall be made aware of EMF prior to the event.

\subsubsection{Social Services}

Given the nature of the event, it is recommended that local social services be made aware of EMF.

\subsection{Major Incidents}

In the event of a major incident occurring on site, all personnel, unless
actively treating a patient are to report to the First Aid Post for
instructions. This includes these first-aid volunteers who are currently on
shift.

All EMF first-aid resources will be placed under the direction of the Ambulance
Service NHS Trust either on site or until their arrival.

Any fatalities will be dealt with by the Police in conjunction with the Event
Management Team and the NHS Ambulance Service in line with the Police Plan for
this eventuality.
