\subsection{Overview}
In accordance with the Health and Safety Executive guidance, EMF camp will
provide, at any one time, 4 first-aiders, available and on call 24 hours a day
for the duration of the festival. Their remit is threefold:

\begin{itemize}
  \item Provision of first aid to any festival attendees, and EMF staff or
  volunteers, throughout the duration of the festival and during setup and
  strike-down
  \item Support for the basic welfare of the festival attendees and EMF
  staff and volunteers
  \item Management of any situations involving lost children for the duration of
  the event (see document - Lost Child Policy)
\end{itemize}

First Aid practice is defined by the 10th Edition of the First Aid Manual
(Published 2014, Dorling Kindersley), the official manual of the Red Cross, St
John’s Ambulance and St Andrew’s Ambulance. First-aid volunteers will be
required to be familiar with this document and operate within its instructions.
A copy of the manual will be available on site for reference.

\subsection{Recruitment}
All first-aiders at EMF are volunteers and must present two different kinds of
credentials to the First Aid Team Lead: Each volunteer must have, as a minimum,
a qualification that meets the guidelines and criteria as defined by the Health
\& Safety Executive (HSE) in respect of the 1981 (First Aid) Regulations, such
as a First Aid at Work certificate.

Those with qualifications that are equivalent to, or superior than,
first-aid-at-work will also be accepted. Examples include:

\begin{itemize}
  \item Healthcare professionals, for example GPs, nurses, or paramedics
  \item Community First Responder
  \item First aider with St John’s Ambulance
  \item First aider with the Red Cross
\end{itemize}

In all cases, a copy of the relevant qualification(s) will be checked by the
First Aid Team Lead before the event. A digital copy will be retained by the
First Aid Team Lead and the event organiser in a secure format in accordance
with the Data Protection Act.

\subsection{Disclosure}
As first aid volunteers may need to work with children or vulnerable adults an
Enhanced Disclosure from the Disclosure and Barring Service is required. Proof
of this must be presented to the First Aid Team Lead and the festival organiser
before the event. These without a current Enhanced Disclosure will be required
to apply for one through the EMF  before the event. Digital copies of disclosure
paperwork for all volunteers will be kept securely by the First Aid Team Lead and
EMF organiser in accordance with the Data Protection Act.

\subsection{Strategy}
First aid cover will be provided 24 hours a day for the days when the festival
is open to the public. In addition, cover will be provided for the setup and
strike-down of the festival as long as EMF staff and volunteers are on site.

Cover will be organised as 3 shifts of 8 hours, with four first aiders on duty
at any one time. First aiders will operate in pairs, with one pair roaming the
site, and the other pair based at the designated first aid point. Both teams
will be provided with radios.

The shifts will commence at 8am in the morning. The midnight to 8am shift will
be covered by two first aiders and will be an 'on-call’ service, i.e a mobile
phone number will be provided to all the EMF personnel and stewards for the
first aiders on duty.

\subsection{Equipment and situation}
Equipment will be provided for each first aid volunteer. These will include a
first aid kit, a high-visibility tabard and a radio. Volunteers will be advised
to bring appropriate warm and waterproof clothing and footwear.

First aid supplies and equipment will be purchased from St Johns Ambulance
supplies. The First Aid Team Lead will be responsible for the first aid
equipment and consumables for the team, and checking that it meets the required
standards.

The first aid point will also operate as a welfare point. It will therefore be
equipped with bottled water, suncream, and well as camp beds and heaters (see
below).

The first aid point will be in a dedicated tent, clearly marked on the site plan, and signposted outside. Its location will be made known to EMF staff, volunteers and attendees and will be staffed from 8am to midnight through-out the event, with the overnight first aid cover provided on an on-call basis.

\subsection{Medical waste disposal}
All medical waste will be disposed of in the correctly marked bags (orange,
clinical waste bags) and will be kept at the first aid point until the end of
the strike-down, whereupon they will be disposed of in accordance with the
local health authority's requirements. Sharps bins will be provided and
disposed of in the same fashion if used.

\subsection{Reporting}
All first aid administered will be recorded electronically on the
‘patient-report-form', with any serious incidents being reported on an
additional RIDDOR form. Patient report forms will be kept by the First Aid Team
Lead and the EMF organiser with a copy sent to the patient upon request. This
data will be kept in a secure format and accordance with the Data Protection
Act.

\subsection{Local Authorities}
Both the local ambulance service control room and the local police service will
be informed, prior to festival, that an event is taking place where first aid
cover is being provided.
