\section{Event Overview}

Electromagnetic Field (EMF) is the UK's largest technology-focused camping
festival: A non-profit three-day event for those with an inquisitive mind and
an interest in science, engineering, technology, crafts, DIY, and computer
security. EMF 2018 is the fourth major camping festival organised by
Electromagnetic Field Ltd.

Held every two years, EMF can be seen as a cross between a conference and a
festival, with talks and workshops on a wide range of subjects across a series
of marquee-based stages. EMF is a uniquely community-run event, with attendees
grouping into "villages" around shared interests. Villages run their own
workshops, talks, and installations in addition to those run by EMF itself.

EMF provides a high-speed internet connection and WiFi to the entire festival.

Our last event in 2016 brought 1900 attendees to a site near Guildford for
talks and workshops on topics ranging from genetic modification to electronics,
blacksmithing to high-energy physics, reverse engineering to lock picking,
computer security to crocheting, and quadcopters to beer brewing.

\subsection{Key Information}

\begin{tabular}{l l}
Public Event Hours & 12:00 Thursday 30\th August -- 17:00 Monday 3\rd September 2018 \\
Maximum Site Capacity & 2500 \\
Targeted Capacity & 2100 \\
Ticket Price & £100--£140 \\
Location & Eastnor Castle Deer Park, Eastnor, Herefordshire \\
\end{tabular}

\subsection{Demographic}

EMF events have historically attracted a broad spectrum of attendees due to the
variety of talks and workshops available. The audience demographic is expected
to range in age between 18 and 50, with the majority of attendees being between
22 and 35, and a slight male bias.

EMF is a family-friendly event with activities for children, and in 2016 there
were a significant number of families attending -- a number which we expect to
improve on again in 2018.

\section{Management}

Electromagnetic Field is an entirely volunteer-run event, with all staff
volunteering signficant amounts of their time to organise the event, and many
attendees volunteering some of their time to make the event happen.

We strive to operate EMF to a standard equivalent to professionally-run events.
Many of the organising team have significant experience at previous EMF events,
as well as several similar events held across Europe.

Overall responsibility for the operation of the event lies with the Directors
of Electromagnetic Field, Russ Garrett and Jonty Wareing.

Those volunteers involved in running the event are formed into teams, with each
team having an experienced lead and a deputy lead who are accountable for that
team.

Organisational meetings of all team leads (the ``organising team'') are held
periodically prior to the event.  During the event, meetings of the organising
team will be held daily to deal with any issues arising.

At all times during the public opening hours of the event, an experienced member
of the organising team will be on call as the ``duty site manager'' with
ultimate responsibility for event operations.

\section{Site}

The event will be located in grassland at Eastnor Castle Deer Park. EMF has an
agreement to use the site between Friday 24\superscript{th} August and Thursday
6\superscript{th} September.

Where not otherwise secure, the site will be surrounded by a Heras-style fence
for security purposes.  This perimeter will surround the event tents and all
camping areas. All licensable activities will happen within this perimeter.

The entrance gate will be staffed 24 hours per day, and tickets will be
exchanged for wristbands on entry.

Further detail of the site layout, including the locations of tents, toilets,
and water supply, can be found on the site plan in Appendix~\ref{site-plan}.

\section{Licensing}

EMF is in the process of applying for a premises license from Herefordshire
Borough Council.

The Designated Premises Supervisor is Russ Garrett.

\subsection{Alcohol Sales}

The bar will be operated by EMF and staffed by volunteers. All bar staff will
receive training on legislation and event policies before starting work. The
bar will operate a ``Challenge 21'' policy for dealing with under-18s, and will
only accept approved documents as proof of age. Bar staff will be instructed
not to serve customers who are drunk, and will be familiarised with the
strength of the drinks they are serving.

Volunteer bar staff will be required to complete training on the legal
responsibilities of alcohol sales prior to undertaking any sale of alcohol
on the premises.

A full price list will be provided at each bar, which will include the ABV
levels of each drink and the measured quantity in which spirits are being sold.

Drinks will not be served in glass containers. Attendees will be requested not to
bring glass onto the site.

Previous events have had no recorded alcohol-related disorder, and we do not
expect this event to be any different.

Alcohol sales for EMF will be overseen by Russ Garrett and Steve Early who
hold personal licenses.

\subsection{Regulated Entertainment}

The main focus of EMF is talks and workshops, however the event will provide
some live music and DJs, as well as ancillary recorded music between talks.
Music is a secondary focus of the event, and will not be a major component of
any promotion or advertising.

EMF will operate a curfew for music performances of 02:00 (23:00 on Sunday),
with performances scheduled to finish 30 minutes earlier to allow for overruns.

Further information on EMF's noise management policy can be found in
section~\ref{noise}.

\subsection{Public Nuisance}

The event site is not in direct proximity to any residential areas.

EMF is a camping festival, and no day tickets will be on sale. The majority of
attendees are not expected to leave the site during the event. Previous events
have had no recorded incidents of public nuisance.

Because of this, we consider that there is a low risk of attendees causing
public nuisance outside the site.

\subsection{Prevention of Crime and Disorder}

An log will be kept at the premises, and made available on request to the Council
or the Police, which will record:

\begin{itemize}
    \item crimes reported to the venue
    \item ejections of attendees
    \item incidents of disorder
    \item visits by a relevant authority or emergency service
    \item changes of Duty Site Manager
\end{itemize}

Attendees will be advised not to leave valuables in their cars, and a secure property
lock-up will be provided on the site.

\section{Noise}
\label{noise}
We are mindful of the need to keep noise nuisance to an absolute minimum and
EMF will cooperate fully with site management, Environmental Health, and local
residents to achieve this.

Programmed music performances will be confined to one of the main tents
(approximately 12x24m in size), and will be oriented to direct sound away from
residential areas. Due to the comparatively small audience size, sound levels
can be kept low and easily controllable.

Staff involved with noise monitoring, the duty site manager, and staff at any
locations within the site using amplified music will be in radio contact and
instructed to effectively and swiftly reduce noise levels if necessary.

Local residents and Environmental Health will be provided with 24/7 contact
details for the event control in case of an issue with noise.

Due to the nature of the event, it is unavoidable that some attendees will
bring sound systems.  The organisers will endeavour to ensure that any
amplified music played by attendees is inaudible at noise-sensitive boundaries
outside of the licensed hours, and sanctions will be imposed to enforce this.

\section{Children and Young People}

The event will be family-friendly with a dedicated children's area. Under-5s will receive free tickets
and under-16s will receive a discount.

All reasonable efforts shall be made to ensure that there are no unaccompanied under-16s on site.

A DBS-checked volunteer will always be on-call in case of a lost child situation. The lost child
policy can be found in Appendix~\ref{lost-child-policy}.

\section{Water}
Mains water is available on site from several standpipes. Taps and washbasins will be provided by EMF.

All temporary water installation will be in compliance with BS8551 and water authority regulations.

In the event of a water supply failure, an emergency agreement will be in place with a water supplier.

\section{Toilets \& Sanitation}

As there are no toilet facilities on site, a minimum of 25 toilets and 5 urinals,
as well as two disabled toilets, will be provided by an external supplier. This
is well in excess of the recommendations made by the Event Safety Guide.

Showers will also be provided.

Greywater will be stored in existing septic tanks on site and as far as is practical
will not be allowed to run onto the ground and into watercourses.

\section{Food}

Food on site will be provided by commercial food concessions. Food hygiene certificates,
risk assessments, and insurance details will be checked and kept on file for all food
concessions, and will be made available to site management and Environmental Health before
the event on request.

\section{Staffing}

As is common with similar events, we aim to provide as many staff as possible
by asking attendees to volunteer. All stewarding will be overseen by an
experienced stewarding co-ordinator.

\subsection{Staffing Levels}

Staff levels will be allocated as follows:

\begin{tabular}{l l l}
Role & Period & Staff \\
\hline
Main Gate & 24/7 & 2 -- 4 \\
Information Point & 24/7 & \\
Roving & 24/7 & 2 -- 4 \\
Bar & Licensed Hours & 2 -- 5 \\
Per Tent & During Talks & 2 \\
Control & 24/7 & 1
\end{tabular}

As well as these staff, a dedicated duty event manager will be on call at all times during the event.

In the unlikely event staffing levels cannot be guaranteed, external stewarding services will be sought.

\section{Communication}

\subsection{Radio}
All key members of staff, including stewards at gates and parking,
will be issued with a radio and a contact list, and will be trained in its use.

A ``control'' member of staff at event control will be contactable by
radio and telephone at all times and will have emergency contact details for the
organisation team.

\subsection{Telephone}
A number of fixed and mobile phones will be available at the event control so
contact telephone numbers can be made available to external parties and staff
who may be out of radio range. Telephone numbers will be easily re-routed to
staff mobile telephones in case of emergency.

The event also has an internal telephone network covering all major stages and
other locations within the event, which can be used for sensitive
communications.

\subsection{Communications to Attendees}\label{attendeecomms}
Public information shall be capable of being broadcast at all stages by the
stage managers. Loud hailers will be available for use by relevant staff to
give information directly to attendees.

As WiFi is available to the entire site at EMF, social media comprises a
significant part of attendee communication, and this will also be used in
emergencies as long as the internet connection is operational.

Communication to attendees before the event will be via email and social
media. Attendees must provide a valid email address when purchasing tickets.

\section{First Aid}

First aid will be provided by EMF's own team of experienced volunteers,
which include those with advanced St John Ambulance qualifications and previous
experience in large-scale festival first aid, as well as practicing doctors.

During setup and teardown there will be at least one qualified first-aider on
duty during working hours. During the licensed period, there will be at least
four qualified first aiders on duty at all times.

Further information on first aid can be found in the first aid policy in
Appendix~\ref{first-aid-policy}.

\section{Traffic Management and Parking}

Attendees will be encouraged to use public transport or car-share as much as
possible. Car parking on site will be ticketed, and vehicles will not be
allowed on site without a pass. A shuttle bus service will be provided between
the site and Ledbury station.

Approximately 800-1000 vehicles are expected to park on site, and the capacity
available for car parking is well in excess of this.

Attendees will be provided with comprehensive directions to the site in advance
of the event.

The vehicle entrance for attendees will be the main deer park entrance, east of
Eastnor on the A438. This will provide approximately 1km of vehicle queueing
capacity on private roads within the deer park. Signs marking the entrance to
the site will be placed on Eastnor's land - no signage is expected to be required
on public highways.

The maximum rate of vehicle arrival is expected to be 60-80 vehicles per hour
for a short period on Friday morning. Sufficient stewarding capacity will be
provided to ensure all vehicles are promptly parked, and no queueing is expected
on public roads.

Accurate directions to the event site will be provided to all attendees in
advance, and signs marking the site entrance will be placed on land owned by
Eastnor Castle. No signage on public roads is expected to be required.

\subsection{On-Site Vehicle Safety}

Vehicle movements within the perimeter fence will be restricted to essential
journeys during peak hours (11:00--23:00) and co-ordinated by radio. No
un-marshalled vehicles will be allowed to move during peak hours.

The main vehicle gate will be separate to the pedestrian entrance gate to
reduce interactions between pedestrians and vehicles.

Catering concessions will be able to access the site via a separate
vehicle gate.

Only competent members of staff will be allowed to use site plant. Their
training and certification will be checked before the event.

\section{Electrical Installations and Lighting}

There is no existing mains electrical supply on site. All electricity will be
supplied from a temporary electrical system provided by EMF.

The temporary electrical system will consist of pre-tested, modular units
provided by a reputable temporary event power supplier. The system will be
installed and tested by EMF staff, with assistance from external contractors
where required.

The electrical system will be installed, tested, and operated by competent
persons in accordance with BS7909:2011 (\textit{Code of practice for temporary
electrical systems for entertainment and related purposes}) and the latest
BS7671 wiring regulations.

All final circuits rated less than 32A will be protected by 30mA, 10ms Residual
Current Devices (RCDs).  All other circuits, except those run through secure
staff-only areas, shall be protected by RCDs with ratings chosen to maximise
protection which keeping nuisance trips to a minimum.

\subsection{Lighting}
General site lighting will be provided by festoon-style lighting along paths,
with floodlights provided where necessary.

Lighting on primary throughfares will be fed from two independent generators
to maintain lighting during the failure of one generator.

\section{Special Effects}
EMF welcomes attendees to bring their own installations and demonstrations.
Some of these installations may employ special effects with their own risks,
and EMF will, where reasonably possible, put straightforward processes in
place so these can be used safely. 

These special effects policies apply to all effects on site, whether provided
by EMF or attendees.

\subsection{Strobe Lighting}
EMF has previously welcomed a number of attendees with photosensitive epilepsy.
Due to the risk to those people, no discharge-style strobe lights will be allowed
on site, on stages, installations, or otherwise.

Those responsible for lighting systems which have the capability to flash brightly
at frequencies capable of triggering photosensitive epilepsy (3-50Hz) will be
informed that such effects are not permitted to be used on site.

\subsection{Lasers}
Unless otherwise authorised by EMF, all laser effects on site will be limited to
Class 1 lasers with no audience scanning or possibility of direct eye exposure.

Laser effects which do not meet the above limits must either be provided and
operated by a reputable contractor, or authorised by a person competent to
evaluate their safety. A permit will then be granted by EMF.

Unauthorised lasers, including laser pointers, must not be used on site.

If laser effects are to be used outside with scanning near or above the
horizon, a NOTAM (Notice to Airmen) will be filed in advance with the Civil
Aviation Authority.

\subsection{Flame Effects}\label{flameeffects}
Flame effects have become an integral part of the atmosphere of EMF, and
we will continue to allow their safe use. The majority of flame effects
at EMF are Liquefied Petroleum Gas (propane) effects.

A full set of rules for flame effects at EMF will be provided to those
who intend to bring them. These rules are based on the comprehensive
Flame Effects Guidelines published by the Burning Man event in the USA,
with reference to the HSE publication SIM 05/2004/09 (Safe use of
liquefied petroleum gas fired stage flame effects).

A full design of the flame effect must be submitted to EMF prior to the
event. EMF will verify the design and construction of flame effects meets
these rules and issue a permit for use once it has arrived on site and
been checked.

These rules include requirements for:
\begin{itemize}
    \item Construction standards to prevent gas leaks
    \item Location of effects away from people and flammable objects
    \item Provision of fire extinguishers
    \item Operating guidelines and supervision
\end{itemize}

Use of flame effects without a permit at EMF will be strictly prohibited.

\subsection{Pyrotechnics}
Pyrotechnic sepcial effects of any type will not be used at EMF.

\section{Contingencies}\label{contingencies}
Contingencies which may require the use of the emergency plan include (in
approximate order of likelihood):

\begin{itemize}
  \item Severe weather
  \item Fire
  \item Major accident/illness
  \item Collapse of structures
  \item Bomb/terrorist threat
\end{itemize}

In the event that an emergency situation develops, procedures will be in place
to ensure that it is promptly escalated to the Duty Site Manager who will assume
overall control of the incident, assisted by other members of the organising
team as necessary.

If an announcement needs to be made to attendees, the incident team is able to
ensure announcements are made via stage PA systems, loudhailers, and online channels.
There is no site-wide PA system at EMF\@. Further information on emergency communications
to attendees can be found in section~\ref{attendeecomms}.

It is the Duty Site Manager's responsibility to ensure that the emergency services
are called if necessary.

\subsection{Major Incidents}
A major incident is any emergency which puts a significant number of people at risk
of harm and/or requires the large-scale assistance of the emergency services.

The decision to declare a major incident will be made by the incident team in
consultation with the emergency services.

Once a major incident is declared, control over staff on site will be transferred
to the emergency services until the major incident has ended.

\subsection{Evacuation}
Due to the size of the site and provision of fire lanes, a full evacuation of the site
is unlikely to be required.

If necessary, announcements will be made to evacuate attendees to another part of the
site, and the incident area cordoned off.

In the unlikely event of a full evacuation being needed, emergency gates will be used
to facilitate fast evacuation.

\subsection{Severe Weather}
Severe weather, including heavy rain, high winds, and lightning, may require the event
programme to be altered or even cancelled.

Prior to and during build-up, weather forecasts will be periodically monitored to ensure
that ground conditions are acceptable. During the event, forecasts will be reviewed
daily for risks of severe weather.

Maximum wind ratings for temporary structures will be requested from suppliers and
easily accessible at event control to ensure that the management team is aware of weather
limitations.

In the event of severe weather being forecast during the event, the organising team will
be informed and weather will be monitored on a regular basis. Attendees will be warned
using the methods in section~\ref{attendeecomms} if necessary.

In the event of severe weather posing a risk to temporary structures due to wind or
lightning, an emergency situation will be declared in accordance with
section~\ref{contingencies}. The entire event programme will be halted and attendees
evacuated from those structures by stage managers with assistance from other stewards
if necessary. Depending on the nature of the severe weather, it may be appropriate to
evacuate attendees to their cars until the weather passes.

If severe weather produces conditions which will not allow the event to safely continue,
the emergency team will begin the event cancellation process detailed in
section~\ref{cancellation}.

During severe weather, especially at night, generators must not be switched off. If
necessary, circuits at risk of potential water exposure should be isolated in
order to preserve power for site lighting and essential services. During severe weather,
Lack of site lighting poses a much more significant risk than electric shock as all
circuits are RCD protected.

\subsection{Cancellation}\label{cancellation}
In some circumstances the event may have to be cancelled before or during the event.

\subsubsection{Before the Event}
If cancellation is required before the event starts, attendees will be informed by
direct email and social media. 

If cancellation happens shortly before the scheduled start of the event, staff will
be made available at the gate to inform any arrivals of the cancellation and ensure
they promptly depart.

\subsubsection{During the Event}
If cancellation is required during or very shortly before the event, those attendees
on site will be informed using the methods in section~\ref{attendeecomms}. Attendees
not yet on site will also be informed.

Once the cancellation process has started, all programmed activities on site will
cease.

In some cases it may be appropriate to keep attendees on site overnight following the
until they can depart safely. If the cancellation is due to severe weather, some
attendees may not have access to their tents, and temporary structures may be repurposed
to accommodate them.

\section{Fire}
\label{fire}
\subsection{Sources of ignition}

The main sources of ignition at EMF are:

\begin{itemize}
\item Hot exhaust from generators
\item LPG appliances in catering area
\item Camp fires, BBQs, and gas appliances used by attendees
\item Flame effects/pyrotechnics
\end{itemize}

\subsection{Steps to minimise risk}
The following steps will be taken to mitigate risks of fire:

\begin{itemize}
\item Generators provided by EMF will be sited away from all combustible materials in accordance with supplier's guidance.
\item No other generators will be allowed on site.
\item Combustible materials will be stored away from structures.
\item All LPG cylinders which are not in use will be stored in a fenced enclosure at least 5m away from any structures.
\item Temporary structures will be sourced from reputable suppliers and have appropriate fire safety certification.
\item Firefighting equipment will be provided on site and staff made aware of its location. Fire extinguishers will be sited in all event tents.
\item Fire lanes will be provisioned within camping areas, clearly marked, and enforced.
\item Camp fires or BBQs will not be allowed on the ground.
\item Attendees will be instructed not to use gas appliances in tents.
\item Roving staff will be instructed to monitor the site for any fire hazards and contact control over radio.
\item Catering/concessions staff will be made aware of regulations regarding gas storage.
\item Catering area will be sited well away from camping area.
\item Sufficient access to the site will be provided and maintained clear for access of fire appliances.
\item Weather conditions will be monitored in case of very dry conditions raising the risk of spread of fire through vegetation.
\item Flame effects will not be authorised for use unless they meet the standards. See section~\ref{flameeffects}
\end{itemize}

All stewards will be briefed on steps to take if a fire is discovered which
will include alerting other staff by radio and, if necessary, evacuating
attendees.


\newgeometry{top=1cm,bottom=1cm,left=1cm,right=1cm}
\begin{landscape}
\thispagestyle{empty}
\section{Event Risk Assessment}

\begin{tabular}{| p{3cm} | l | p{1.5cm} | p{9cm} | p{1.5cm} | p{2cm} | p{6cm} |}
\hline
\textbf{Hazard} & \textbf{Risk} & \textbf{Affected Parties}
& \textbf{Control Measures} & \textbf{Resulting Risk} & \textbf{Responsible Team} & \textbf{Comments} \\ \hline

Electric Shock & Moderate & Everyone &
All electrical installations to conform to BS7671. 30mA RCDs on all final circuits.
Regular visual checks. Attendees who require power should be made aware of the risks. &
Low & Power & \\ \hline

Fire & Moderate & Everyone & Please refer to section~\ref{fire} for the fire risk assessment. &
Low & All & \\ \hline

Injury from vehicles operating on site & Moderate & Everyone &
Vehicle movements on site to be restricted during peak hours
(11:00--23:00) and co-ordinated by radio.
5mph speed limit to be enforced on site.
No un-marshalled vehicles during peak hours.
Site plant to only be used by appropriately experienced persons. &
Low & Stewards & \\ \hline

Falls from Heights & Moderate & Staff &
Any work at heights more than 3m should only be carried out in accordance with an
additional risk assessment.
Access equipment must only be used by those who are trained to use it.
& Low & Setup & \\ \hline

Trips \& Falls & Moderate & Everyone &
As far as is practical, ensure all cables are buried or flown above head height. Ensure site is adequately lit. &
Moderate & Setup & Trip hazards (guy ropes, etc.) will always be present on a camp site.\\ \hline

Food Poisoning & Moderate & Everyone &
All food concessions to have food hygeine certificates checked and on file.
Water supply to be installed and tested in accordance with regulations. Adequate handwashing facilities
to be made available.
& Low & Vendors &\\ \hline

Glass injuries & Moderate & Everyone &
Discourage bringing glass onto site. Alcohol should be served in plastic or paper cups. &
Low & Stewards & \\ \hline

Crowd Safety/Crushing & Low & Everyone &
Stewards to monitor situation and report by radio. &
Low & Stewards & Event has historically been low-energy. \\ \hline

Injury from collapse of temporary structures & Low & Everyone &
All temporary structures are provided and erected by reputable external contractors.
Certificates of completion will be kept on file.
Wind speed to be monitored and structures closed if design limits are exceeded. &
Low & Setup & \\ \hline

Dehydration \& Sunburn & Low & Everyone &
Water readily available. First aiders on site. If weather is very warm, remind attendees to drink water.
& Low & First Aid & \\ \hline

Insect bites \& stings & Low & Everyone &
First aiders on site. & Low & First Aid & \\ \hline

\end{tabular}
\newpage
\thispagestyle{empty}
\section{Buildup and Teardown}
\subsubsection{Responsibility}
The table below shows the division of responsibility between EMF and external contractors
for major setup and teardown tasks:

\begin{table}[h!]
\begin{tabular}{| p{10cm} | l |}
\hline
\textbf{Activity} & \textbf{Responsible} \\ \hline
Laying of fencing & EMF \\
Temporary electrical installation & EMF \\
Rigging and setup of staging/AV/lighting/sound equipment & EMF \\
Rigging of decorative and emergency site lighting & EMF \\
General transportation of items around site, manual handling & EMF \\
Laying of temporary trackway & External Contractors \\
Buildup of temporary structures & External Contractors \\
\hline
\end{tabular}
\end{table}

\subsubsection{Risk Assessment}

\begin{tabular}{| p{3cm} | l | p{10cm} | p{3cm} | p{6cm} |}
\hline
\textbf{Hazard} & \textbf{Risk} &  \textbf{Control Measures} & \textbf{Resulting Risk} & \textbf{Comments} \\ \hline
Injury from vehicles & Medium &
EMF staff on site will be required to wear high-vis jackets while walking around site.
15mph site speed limit.
Restrict vehicle movements between sunset and sunrise. 
Brief staff on site in how to treat construction vehicles.& Low & \\ \hline
Falls from Height & Moderate &
Any work at heights more than 3m should only be carried out in accordance with an
additional risk assessment.
MEWPs to be used instead of ladders where appropriate.
Access equipment must only be used by those who are competent to use it.
& Low & \\ \hline
Injury from collapsing structures & Medium &
Temporary structures to be constructed by contractors only.
EMF staff to be kept clear until structures are signed off.
    & Low & \\ \hline
Electric shock & Low &
Reputable suppliers used for electrical equipment.
Visual inspection before deployment of equipment.
Distribution equipment to only be powered on by competent people after
inspection and testing in accordance with BS7909.
    & Low & \\ \hline
Foot/leg injury & Medium &
Staff informed about appropriate footwear in advance.
Staff reminded of the dangers of stepping off moving vehicles, even at low speed.
    & Low & \\ \hline
Hand injury & Medium &
Gloves to be provided by EMF to reduce hand injuries during manual handling.
    & Low & \\ \hline
\end{tabular}


\end{landscape}
\restoregeometry

\appendix

\section{First Aid Policy}
\label{first-aid-policy}
\subsection{Overview}
EMF will provide first aiders 24 hours per day for the duration of the festival.
Their remit is threefold:

\begin{itemize}
    \tightlist
  \item Provision of first aid to festival attendees and EMF staff throughout
      during the event and during setup and strike-down
  \item Support for the basic welfare of the attendees and staff
  \item Coordination of situations involving lost children, in conjunction
      with the external childcare provider.
\end{itemize}

\subsection{Responsibility}
Responsibility for first aid at EMF lies with the first aid team leads.

\subsection{Risk Assessment}
On basis of the nature of the event and our experience of in previous years,
the risk of medical presentations is low/medium.

\begin{table}[h!]
    \caption{Previous event statistics}
    \label{table:firstaiddata}
    \centering
    \begin{tabular}{| l c c c c |}
        \hline
            & & & \multicolumn{2}{c |}{\textbf{Patients taken to hospital}} \\
            \textbf{Year} & \textbf{Capacity} & \textbf{First Aid Incidents} &
            \textbf{by car} & \textbf{by ambulance} \\
        \hline
            2018 & 2350 & 111 & 0 & 0 \\
            2016 & 1900 & 56 & 2 & 1 \\
            2014 & 1100 & 65 & 2 & 0 \\
            2012 & 499 & Not Recorded & 0 & 1 \\
        \hline
    \end{tabular}
\end{table}

The event is low-energy and is not comparable to a similar-sized music festival
in terms of first aid requirements.

Previous events have had a low incidence of patients needing hospital care (0.15\% of
attendees in 2016 and none in 2018). \Cref{table:firstaiddata} shows the historical
first aid statistics for the last three events. It should be noted that ``First Aid
Incidents'' includes all interactions with the first aid team, including minor cuts,
blisters, and bites.

\begin{table}[h!]
    \caption{Event Resource Assessment calculation}
    \label{table:narucalc}
    \centering
    \begin{tabular}{| r l l c |}
        \hline
           & \textbf{Item} & \textbf{Details} & \textbf{Score} \\
        \hline
        A & Nature of event & Music Festival (closest match) & 3 \\
          B & Venue & Includes overnight camping & 5 \\
          C & Standing/Seated & Mixed & 2 \\
          D & Audience Profile & Full mix, not in family groups & 3 \\
          E & Past History & Good data, previous casualty rate <1\% & -1 \\
          F & Expected numbers & < 3000 & 2 \\
          G & Expected queueing & Less than 4 hours & 1 \\
          H & Time of year & Summer & 2 \\
          I & Proximity to definitive care & More than 30min by road & 2 \\
          J & Profile of definitive care & Choice of A\&E departments & 1 \\
          K & Additional hazards & None & 0 \\
          L & Additional on-site facilities & None & 0 \\
          \textbf{Total} & & & \textbf{20} \\
        \hline
    \end{tabular}
\end{table}

Using the Event Resource Assessment guidelines from NARU~\cite{naru} (formerly
the Purple Guide guidelines on medical provision), an initial assessment of the
required medical resources can be made in \cref{table:narucalc}.

This assessment places EMF at the upper edge of the lowest category of resource
requirement, requiring 4 first aiders only.

\subsection{Staffing}

The staffing levels for the first aid team are shown in \cref{table:fastaffing}.
First aiders will carry out 8-hour shifts during the event, while paramedics
and HCPs will be on call.

\begin{table}[h!]
\caption{Staffing Levels}
\label{table:fastaffing}
\centering
\begin{tabular}{| l l l |}
    \hline
        \textbf{Type} & \textbf{Number} & \textbf{Availability} \\
    \hline
        First Aider & 4 & Roaming site or at medical post (10:00--02:00) / on call (02:00--10:00) \\
        First Aider & 2 & On call \\
        Paramedic & 1 & On call 24/7 \\
        HCP & 1 & On call 24/7 \\
    \hline
\end{tabular}
\end{table}

The risk of volunteer first aiders not turning up for shifts is historically low due
to the nature of the event. In the event of a first aider not turning up for a shift,
other first aiders, or the on-call Paramedic or HCP, will be available to cover.

In the unlikely event of there not being enough staff available to carry out these
staffing levels, the issue will be escalated to event management who will make the
decision to cancel the event unless suitable cover can be promptly arranged.

\subsection{Ambulance}

Due to the distance from nearby hospitals, an ambulance from a CQC-registered provider 
will be available on site at all times when site occupancy is expected to be more than
1000 persons, in order to transfer patients to a hospital if needed. The ambulance will be parked
adjacent to the First Aid tent, and will have access to the clear route off site via gate
X and access route A.

\subsection{Strategy}
The event medical facility will be phased in as attendees arrive. This facility will
be open between the hours of 10:00 and 02:00 over the course of the event. Outside
of these hours, an out-of-hours number will be available to call an on-call first aider.
This will be advertised to all attendees and on-duty staff will be made aware.

This first aid point will be equipped with Basic Life Support (BLS) equipment including
an AED, oxygen therapy, and medical interventions. The first aid point will also operate as
a welfare point, equipped with bottled water, sun cream, camp beds and heaters.

A registered paramedic will be on site in possession of a full ALS kit, including advanced
interventions. This paramedic will be on-shift for 12 hours per day during the event. At
night, they will maintain an on-call presence using telephone contact.

HCPs (Doctors, Paramedics and Nurses) will be rotated through on-call cover during the
event. An on-call HCP will be available at all times to handle any adverse incidents.

Responder teams consisting of two first aid trained members of staff will patrol the event
during opening hours. They will patrol with a large BS 8599-1 first aid kit and will receive
backup of advanced BLS equipment as required.

Pre-cover for the event will be undertaken by on-site event staff who have undertaken first aid
at work qualifications with additional qualifications which include working from height, plant
and electrical safety.

\subsection{Access}

Access to the site is via road. The first aid point will be located near the
entrance to the site via access route A. In the event that an ambulance has
to enter or leave the site, the first aid team will liaise with the stewards to ensure
the area of road around the site entrance is cleared.

Non-urgent casualties who require hospital treatment but not an ambulance will
be taken to the receiving hospital via car, either by a car belonging to the
first-aid team, or a taxi. In both cases, the casualty will be escorted by a
member of the first-aid team, if requested to do so by the casualty or if the
first-aider deems it necessary.

The site itself is a mixture of fields and tracks. There are
trackways throughout the site, with extra fire lanes marked for emergency
vehicles. In the event an ambulance is required at another part of the site,
the first-aid team will liaise with the stewards to direct the vehicle and
provide clear access.

\subsection{Communications}

The First-aid team will work closely with the volunteer, info-desk and childcare
teams.

Volunteers and stewards will be made aware of the location of the first-aid
point and the direct phone number for the first aid team. A single number will 
call all first aiders on duty. Volunteers will also be able to use a radio to 
contact first aiders.

Volunteers and members of the general EMF team who respond to a first-aid
situation should advise the casualty to make their way to the first aid point
if they are able to do so. If not, the volunteer should contact the first aid
team immediately and stay with the casualty until members of the first aid team
arrive.

Between the hours of 02:00 and 10:00, the first-aid team will be available via the
first aid phone number. Volunteers responding to first aid situations at this
point should proceed as above, but call the first aid phone number
immediately.

Both the first aid point and the roaming team will be provided with a radio.
Radio protocols, such as channels, codewords and such will be circulated at the
initial briefing, on-site. Particular attention is given to the lost and found
child reporting (see Lost Children Policy).

\subsection{Hospitals}

The nearest hospitals with A\&E departments to Eastnor Castle Deer Park are:

\begin{itemize}
    \tightlist
\item Gloucestershire Royal Hospital, Gloucester: 13.2 miles
\item Worcestershire Royal Hospital, Worcester: 13.7 miles
\item The County Hospital, Hereford: 14.3 miles
\end{itemize}

\subsection{Air Ambulance}

A clear field will be kept for air ambulance landings in the Deer Park south of
the event, in an area which is believed to be an existing HEMS secondary landing
site.

\subsection{Equipment}
Equipment will be provided for each first aid volunteer. These will include a
first aid kit, a high-visibility tabard and a radio. Volunteers will be advised
to bring appropriate warm and waterproof clothing and footwear.

\subsection{Waste disposal}
Clinical waste will be disposed of in the correctly marked bags and
removed by the ambulance contractor. Sharps bins will be provided and
disposed of in the same fashion if used.

Standard waste will be disposed of using the facilities provided by the sanitation team.

\subsection{Qualifications}
Volunteer first aiders must have, as a minimum, a First Response Emergency Care
level 3 (FREC3) or First Person on Scene (FPOS) qualification.

A copy of the qualification will be checked by the First Aid Team Lead
before the event. A digital copy will be retained in a secure format in accordance
with the Data Protection Act.

\subsection{Safeguarding}

As EMF is not a first aid organisation, there is no legal requirement for
volunteers to be DBS checked under the Safeguarding Vulnerable Groups Act 2006
(as amended), with respect to either children (Schedule 4, 2(3B)(1)) or vulnerable
adults (Schedule 4, 7(3A)).

EMF will endeavour to obtain DBS checks for all first aid volunteers. Registered
health care professionals will be assumed to be adequately checked by virtue of
their professional registration.

In the event that a volunteer cannot provide a DBS check due to inadequate time
available, they will be paired with another person for all first aid duties.
This will only apply when a DBS check is clearly unable to be completed due to
lack of time available, and not when the volunteer has had adequate time to
facilitate the check.

Notwithstanding the above, first aiders will work in pairs whenever possible, and
only DBS-checked volunteers or healthcare professionals will be permitted to work
alone, for example when on call overnight.

Trained childcare staff are also available on site as part of the youth area.

\subsection{Insurance}
EMF will carry medical malpractice/indemnity insurance cover for all first aiders.

\subsection{Reporting}
All first aid administered will be recorded on an electronic or paper
patient report form, with any serious incidents being reported on an
additional RIDDOR form. Patient report forms will be kept securely by the First
Aid Team Lead, with a copy sent to the patient upon request. This
data will be kept in a secure format and accordance with the Data Protection
Act.

These members of the first aid team who are members of St John, Red Cross or another
organisation who have their own paperwork requirements may fill out their own paperwork
in addition to EMF's requirements.

\subsection{Local Authorities}
Both the local ambulance service control room and the local police service will
be informed, prior to festival, that an event is taking place where first aid
cover is being provided.

\subsubsection{Social Services}

Given the nature of the event, it is recommended that local social services be
made aware of EMF.

\subsection{Major Incidents}

In the event of a major incident occurring on site, all personnel, unless
actively treating a patient, are to report to the First Aid Post for
instructions. This includes these first aid volunteers who are currently on
shift. The first aid lead on duty should immediately report the situation to the
Duty Site Manager.

All EMF medical resources will be placed under the direction of the Ambulance
Service on request.

Any fatalities will be dealt with by the Police in conjunction with the Event
Management Team and the NHS Ambulance Service in line with the Police Plan for
this eventuality.

\newpage

\section{Lost Child Policy}
\label{lost-child-policy}
This policy document has been prepared for the guidance of everyone working as
part of the volunteer team at EMF and follows Home Office and Department of
Health recommendations. It is essential that all team members adhere to these
guidelines. In the event of a query, team members are advised to consult the
team co-ordinator or her assigned deputy or the appropriate shift leader for
further guidance.

These guidelines are intended as a practical framework for people working with
children in voluntary settings to help ensure the safety, well being and
protection of children in their charge.

It is the responsibility of every member of the EMF volunteer teams to prevent
the physical, sexual and psychological abuse or neglect of children and young
people, or vulnerable adults, in our care and to report any such abuse that may
be suspected or discovered.

The Lost and Found Children service will be provided 24 hours a day while
ticket holders are onsite. All enquiries and dealings regarding lost and found
children will be co-ordinated by the EMF First Aid Team and all staff onsite
will be briefed about this.

The EMF First Aid Tent is the designated lost child point and will be marked as
such on any maps, printed or online EMF information.

\subsection{Reporting Protocols}

Upon receiving a report of a missing/lost child, young person or vulnerable
adult, staff will notify HQ as soon as practicable. HQ will forward this
information on to the first aid team, either via radio (between the hours of
8am and midnight) or via mobile phone (between the hours of midnight and 8am).

All staff at EMF should be made aware as soon as possible, noting the
caveat concerning radios below. All staff on gates to the site should not allow
any child to leave the site until it has been confirmed with the First Aid Team
that the child is not reported lost. Announcements should be made at each
stage. These announcements will be treated as a priority and will be broadcast
at the earliest opportunity. Announcements will not refer to the child
specifically or give personal details, descriptions or names.

Found children should be reported to HQ in a similar manner. In addition, upon
finding a lost child or vulnerable adult, the volunteer in question should make
immediate steps to bring another volunteer to the scene as quickly as
practicable, if they are on their own. It is essential that a lost child or
vulnerable adult not be left in the care of one person. A pair of first aiders
from the will be dispatched and visit the scene in order to escort the child or
vulnerable adult to the EMF First Aid Tent.

While in the care of the EMF First Aid Team, every effort will be made to
ensure the comfort, safety and well-being of the child, young person or
vulnerable adult in a manner which does not violate their human rights and
follows the recommended guidance. Efforts will be made to re-unite the
individual with their parent or guardian, as appropriate, or referral to
statutory agencies as appropriate.

It should be noted that the EMF First Aid Team has no right to detain any
person -- child or considered vulnerable adult -- against his or her wishes.
Efforts will be made to negotiate the best course of action for that
individual.

If there is any suspicion of abuse or neglect of the child or vulnerable adult,
the EMF First Aid Team Leader or Deputy Team Leader must be informed and a
decision will be taken whether to involve the relevant services, such as the
Police \& Social Services.

Time scales will be taken into consideration. If a child or vulnerable adult is
not found within a reasonable time, or a found child is not re-united with a
parent or guardian within a certain time, local authorities will be contacted,
and the situation escalated.

Any individual who is behaving, or expressing a serious intention to behave, in
a manner likely to harm themself or others should be considered at risk.
Support from security and/or Police may be needed while the situation is
assessed.

Any parent/guardian of a child or young person, or friend of a missing person,
who reports them missing may need support and it is to be expected that the
member of EMF staff will direct them, or escort if necessary, to the EMF First
Aid Tent. They may be considerably distressed. At this point, staff should keep
details minimal when notifying the EMF First Aid Team; the team will take full
details.

When a child is reunited with their parent or guardian, identification should be
requested and recorded. Only in extreme circumstances should a child be
allowed to leave without the parent providing some form of ID. Should the child
seem afraid or unwilling to accompany the parent or guardian then assistance
from the Police should be sought. Equally, should the parent or guardian
seem in any way unfit to care for that child then assistance from the
Police may be sought.

\subsection{Radio Usage}
All efforts will be made to restrict the amount of information given over the
radio, such as names or other identifying details. A fixed or mobile phone line
should be used wherever possible. Radio 'code words' for children will be in
use at this event, and all staff with radios will be briefed on these.

\subsection{Definitions and Key Terms}

The Children's Act (1989) defines a child as any person under the age of 18
years. For practical considerations at events such as this, each young person
will be assessed on a case by case basis with regards to the safely and well
being of a minor.

The definition of a vulnerable adult is given in the 'No Secrets' guidelines
published by the Department of Health in 2000 as someone "who is or may be in
need of community care services by reason of mental or other disability, of age
or illness; and who is or may be unable to take care of him or herself, or
unable to protect him or herself against significant harm or exploitation"
Further, it defines abuse as "a violation of an individual's human and civil
rights by any other person(s)."

In particular, it should be noted here that adults (i.e. those aged 18 years or
higher) have the right to make their own decisions unless there are clear
grounds to override this as a result of their lack of capacity OR if wider
public interest is involved.

The law in relation to adults offers far fewer opportunities or
responsibilities in relation to intervention. The principle here is to promote
negotiation with regard to the individual's capacity at that time.

It is essential that the boundaries of confidentiality are explained to the
child or young person or vulnerable adult - if possible before disclosure, i.e.
where it is suspected they might be about to disclose. Under the Children's Act
(1989), we have a duty to inform Social Services of any reports of abuse
involving children and cannot therefore keep such details confidential. This is
for the protection of the individual and possibly others. It is the role of the
team co-ordinator or her assigned assistant to liaise with Social Services in
this matter and she is responsible for making them aware of the disclosure.

Written notes will be kept of all relevant information. Information should
however only be shared on a strictly 'need to know' basis.

\pagestyle{empty}
\newgeometry{top=1cm,bottom=1cm,left=1cm,right=1cm}
\begin{landscape}
\section{Site Location \& Plan}
\label{site-plan}
\subsection{General Location}
\includegraphics[width=23cm]{./supplementary/wide-map.png}
\end{landscape}
\subsection{Site Plan}
[A full detailed site plan will be inserted here in the final version.]
\newpage
\section{Key Contact Details}
\label{key-contact}
[Contact details for key staff will be inserted here in the final version.]
