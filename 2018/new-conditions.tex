\documentclass[a4paper, hidelinks]{article}

\usepackage[T1]{fontenc}
\usepackage[scaled]{helvet}
\usepackage{amsmath}
\usepackage{booktabs}
\usepackage{fancyhdr}
\usepackage{graphicx}
\usepackage{pdflscape}
\usepackage{xhfill}
\usepackage{tabu}
\usepackage{pdfpages}
\usepackage[labelfont=bf]{caption} 
\usepackage[margin=2.5cm]{geometry}
\usepackage[nottoc]{tocbibind}
\usepackage[mathletters]{ucs}
\usepackage[utf8x]{inputenc}
\usepackage[breaklinks=true,unicode=true]{hyperref}
\usepackage[compact]{titlesec}
\usepackage{cleveref}

\renewcommand*\familydefault{\sfdefault}
\newcommand{\superscript}[1]{\ensuremath{^{\textrm{#1}}}}
\newcommand{\subscript}[1]{\ensuremath{_{\textrm{#1}}}}

\setlength{\parindent}{0pt}
\setlength{\parskip}{6pt plus 2pt minus 1pt}
\setcounter{secnumdepth}{2}
\setcounter{tocdepth}{2}

\providecommand{\tightlist}{%
  \setlength{\itemsep}{0pt}\setlength{\parskip}{0pt}}

%% Pad tables out a bit
\renewcommand{\arraystretch}{1.25}

\pagestyle{fancy}
\rhead{}
\chead{}
\lhead{}


\begin{document}
\begin{center}
{\Large Electromagnetic Field 2018} \\
{\large Proposed Premises License Conditions} \\
02/11/2017
\end{center}
\hrule
\section{General}\label{general}

\subsection{Capacity}\label{capacity}

The total capacity of the licensed site will not exceed 2500 persons.

\subsection{Event Management Plan}\label{event-management-plan}

The first draft of the Event Management Plan (EMP) will be produced at
least 60 days (or such lesser period as agreed with the Safety Advisory
Group) prior to the first day of the event and will be submitted to the
Licensing Authority and all the Responsible Authorities along with
detailed site plans that comply with the requirements of the Licensing
Act 2003.

The EMP must be provided to the satisfaction of the Safety Advisory
Group. The EMP will be a working document providing details on how the
event is to be safely and efficiently conducted, informed by the Event
Risk Assessment and Licensing Objectives.

The final EMP will be submitted to the Licensing Authority and all the
Responsible Authorities 14 days prior to the start of the Event. This
must meet with the agreement of all the Responsible Authorities. No
change shall take place to this document without the consent of the
Licensing Authority.

The event must take place in accordance with the Final EMP produced for
the event following approval by the Licensing Authority.

The EMP will include an event risk assessment and provide specific
details on the following areas where applicable:

\begin{itemize}
\tightlist
\item Health and Safety Responsibilities
\item Venue and Site Design
\item Fire Safety
\item Major Incident Planning (Emergency planning)
\item Communication
\item Crowd Management (including steward and security numbers and their roles)
\item Transport
\item Management Structures
\item Barriers
\item Electrical Installations and Lighting
\item Food and Alcohol
\item Water
\item Merchandising and Special Licensing
\item Amusements
\item Attractions and Promotional displays
\item Sanitary Facilities
\item Waste Management
\item Sound: Noise and Vibration
\item Special Effects, Fireworks and Pyrotechnics
\item Camping
\item Facilities for People with disabilities
\item Medical, Ambulance and First Aid Management
\item Information and Welfare
\item Children (including Lost Children Policy)
\item Performers
\item TV and Media
\end{itemize}

Depending on prevailing conditions, the area may be moved to an adjacent
area if deemed necessary to meet the licensing objectives. The layout
will remain the same but the EMP will be amended to take account of the
change.

The Designated Premises Supervisor (DPS) or a nominated deputy (in writing) must be
on the licensed premises and available to the Responsible Authorities
and the Licensing Authority at all times when Licensable Activities are
taking place.

Details of DPS or his deputy who
is on duty when licensable activities are undertaken shall be recorded,
on this premises, at the time. These records shall be made available to the
Licensing Authority or a Responsible Authority on demand. Such record
shall be kept for a period of 12 months after the end of licensable
activities. This information post event must be provided to the
Licensing Authority or a Responsible Authority within 24 hours of the
request.

The name and contact details of the DPS's deputy(s)
will be provided to the Safety Advisory Group in writing no later than 7
days prior to the event.

\section{Prevention of Crime and Disorder}

All staff engaged in the sale of alcohol shall be trained on the legal
responsibilities of alcohol sales prior to undertaking any sale of alcohol
on the premises.

\section{Public Safety}

\subsection{Event Safety Coordinator}

The Premises Licence Holder will appoint an Event Safety Co-ordinator
who will be able to authorise and supervise safety measures.

The Event Safety Co-ordinator will be responsible for:

\begin{itemize}
\tightlist
\item
  Monitoring of contractors
\item
  Liaison with contractors
\item
  Checking of method statements and risk assessments
\item
  Preparation and monitoring of site rules
\item
  Safety inspections and audits
\item
  Collection and checking of completion certificates
\item
  Communication of safety information to contractors and employees
\item
  Monitoring and coordinating safety performance
\item
  Coordinating safety in response to a Major Incident
\item
  Liaison with nominated officers from Herefordshire Council
\end{itemize}

Temporary electrical wiring and distribution systems shall be signed off
in accordance with BS7909:2011 by a competent person prior to any
licensable activity taking place at the premises. The sign off
certificates shall be kept on the licensed premises during the event and
shall be produced for inspection on demand of an `authorised person' (as
defined by Section 13 of the Licensing Act 2003).

The Premises Licence Holder will ensure that all temporary structures
have been inspected and signed off as being safe prior to the
commencement of their use. A copy of each safety sign off certificate
shall be kept on the licensed premises during the event and shall be
produced for inspection on demand of an `authorised person' (as defined
by Section 13 of the Licensing Act 2003). In this condition the term
`temporary structure' means any structure which could cause injury to
someone if it collapsed.

\subsection{Capacity limits}

The maximum permitted numbers in each enclosed structure where regulated
entertainment activities are to be provided will be submitted and agreed
by the Responsible Authorities prior to the commencement of the event.

The premises licence holder must submit arrangements to the satisfaction
of the Responsible Authorities on how such maximum occupancy capacities
will be managed.

All enclosed structures that the public have entry to will have
designated entry/exit points. The numbers of these entry/exit points
will be determined by the occupancy capacity and will be listed in the
Tent Exit Calculation document submitted in the EMP\@. In this licence
``enclosed structures'' are classed as a structure whether tented or
not, which has less than 25\% of its sides open to the atmosphere
(excluding entrances and exits) and which are accessible to the public
when regulated entertainment is provided.

\subsection{Lanterns}\label{lanterns}

Paper lanterns will not be sold on site and will be listed within the
ticketing terms and conditions as items that may not be brought to the
venue.

\subsection{Sanitary Facilities}\label{sanitary-facilities}

A Sanitation Management Strategy will be provided to the satisfaction of
Herefordshire Council's Environmental and Trading Standards Service at
least 60 days prior to commencement of each Event. Once agreed, the
strategy will be implemented throughout the Festival.

\subsection{Water}\label{water}

A Water Management Strategy for the provision of drinking water will be
provided to the satisfaction of Herefordshire Council's Environmental
and Trading Standards Service at least 60 days prior to commencement of
each Event. Once agreed, the strategy will be followed throughout the
Event. No significant changes will be made without consultation with
Herefordshire Council.

\section{Prevention of Public Nuisance}\label{the-prevention-of-public-nuisance}
A Noise Management strategy as approved by Herefordshire Council must be
provided at least 60 days before the commencement of each Event. The
Premises Licence Holder must comply with the Noise Management Strategy.

The Noise Levels (NL) from the event must not exceed the following noise
levels when measured at the designated noise sensitive locations:

\begin{tabular}{l l}
  Time & Limit \\ \midrule
  09:00--00:00 & 55dB \\
  00:00--02:00 & 45dB \\
  02:00--05:00 & 40dB
\end{tabular}

Noise levels should be measured as $dB\, LA_{eq}$, with an averaging period of
five minutes, in a free field position (defined as 3.5 metres
from any reflecting surface other than the ground at a height of 1.2 to
1.5 metres).

The noise sensitive locations are:

  \begin{itemize}
  \tightlist
  \item
    Clenchers Mill Lane, Eastnor
  \item
    Valentines Cottage, Hollybush
  \item
    Caves Folly Nursery, Colwall
  \item
    Hancocks Lane, Little Malvern
  \item
    Rose Mead, Evendine
  \end{itemize}

Or at any other location notified by Herefordshire Council to the
premises licence holder or his deputy.

In addition between 02:00 and 09:00 Thursday to Sunday `noise' from the
event should not be `audible or discernable' within any occupied
permanent structure where people normally reside or sleep, when assessed
with windows and doors closed. In these conditions:
\begin{itemize}
    \item `Noise' is defined as sound which is created by entertainment
          consisting of either vocal (recorded or live) or instrumental
          music (recorded or live) or a combination of both.

    \item `Audible or discernable' is defined as `noise' which is distinct
          above the general hubbub of activity on the site which can be identified
          by the human ear as originating from discrete sources from the licensed
          site.
\end{itemize}


Any testing of sound equipment will not take place before 09:00 and
will last for no more than 2 hours on any one day.

Prominent, clear and legible signage (in not less than 32 font bold)
shall be displayed at all exits to the premises requesting the public to
respect the needs of local residents and to leave the premises and the
area quietly.
\end{document}
