\subsection{Overview}
EMF will provide first aiders 24 hours per day for the duration of the festival.
Their remit is threefold:

\begin{itemize}
    \tightlist
  \item Provision of first aid to festival attendees and EMF staff throughout
      during the event and during setup and strike-down
  \item Support for the basic welfare of the attendees and staff
  \item Coordination of situations involving lost children, in conjunction
      with the external childcare provider.
\end{itemize}

\subsection{Responsibility}
Responsibility for first aid at EMF lies with the first aid team leads.

\subsection{Risk Assessment}
On basis of the nature of the event and our experience of in previous years,
the risk of medical presentations is low/medium.

\begin{table}[h!]
    \caption{Previous event statistics}
    \label{table:firstaiddata}
    \centering
    \begin{tabular}{| l c c c c |}
        \hline
            & & & \multicolumn{2}{c |}{\textbf{Patients taken to hospital}} \\
            \textbf{Year} & \textbf{Capacity} & \textbf{First Aid Incidents} &
            \textbf{by car} & \textbf{by ambulance} \\
        \hline
            2018 & 2350 & 111 & 0 & 0 \\
            2016 & 1900 & 56 & 2 & 1 \\
            2014 & 1100 & 65 & 2 & 0 \\
            2012 & 499 & Not Recorded & 0 & 1 \\
        \hline
    \end{tabular}
\end{table}

The event is low-energy and is not comparable to a similar-sized music festival
in terms of first aid requirements.

Previous events have had a low incidence of patients needing hospital care (0.15\% of
attendees in 2016 and none in 2018). \Cref{table:firstaiddata} shows the historical
first aid statistics for the last three events. It should be noted that ``First Aid
Incidents'' includes all interactions with the first aid team, including minor cuts,
blisters, and bites.

\begin{table}[h!]
    \caption{Event Resource Assessment calculation}
    \label{table:narucalc}
    \centering
    \begin{tabular}{| r l l c |}
        \hline
           & \textbf{Item} & \textbf{Details} & \textbf{Score} \\
        \hline
        A & Nature of event & Music Festival (closest match) & 3 \\
          B & Venue & Includes overnight camping & 5 \\
          C & Standing/Seated & Mixed & 2 \\
          D & Audience Profile & Full mix, not in family groups & 3 \\
          E & Past History & Good data, previous casualty rate <1\% & -1 \\
          F & Expected numbers & < 3000 & 2 \\
          G & Expected queueing & Less than 4 hours & 1 \\
          H & Time of year & Summer & 2 \\
          I & Proximity to definitive care & More than 30min by road & 2 \\
          J & Profile of definitive care & Choice of A\&E departments & 1 \\
          K & Additional hazards & None & 0 \\
          L & Additional on-site facilities & None & 0 \\
          \textbf{Total} & & & \textbf{20} \\
        \hline
    \end{tabular}
\end{table}

Using the Event Resource Assessment guidelines from NARU~\cite{naru} (formerly
the Purple Guide guidelines on medical provision), an initial assessment of the
required medical resources can be made in \cref{table:narucalc}.

This assessment places EMF at the upper edge of the lowest category of resource
requirement, requiring 4 first aiders only.

\subsection{Staffing}

The staffing levels for the first aid team are shown in \cref{table:fastaffing}.
First aiders will carry out 8-hour shifts during the event, while paramedics
and HCPs will be on call.

\begin{table}[h!]
\caption{Staffing Levels}
\label{table:fastaffing}
\centering
\begin{tabular}{| l l l |}
    \hline
        \textbf{Type} & \textbf{Number} & \textbf{Availability} \\
    \hline
        First Aider & 4 & Roaming site or at medical post (10:00--02:00) / on call (02:00--10:00) \\
        First Aider & 2 & On call \\
        Paramedic & 1 & On call 24/7 \\
        HCP & 1 & On call 24/7 \\
    \hline
\end{tabular}
\end{table}

The risk of volunteer first aiders not turning up for shifts is historically low due
to the nature of the event. In the event of a first aider not turning up for a shift,
other first aiders, or the on-call Paramedic or HCP, will be available to cover.

In the unlikely event of there not being enough staff available to carry out these
staffing levels, the issue will be escalated to event management who will make the
decision to cancel the event unless suitable cover can be promptly arranged.

\subsection{Ambulance}

Due to the distance from nearby hospitals, an ambulance from a CQC-registered provider 
will be available on site at all times when site occupancy is expected to be more than
1000 persons, in order to transfer patients to a hospital if needed. The ambulance will be parked
adjacent to the First Aid tent, and will have access to the clear route off site via gate
X and access route A.

\subsection{Strategy}
The event medical facility will be phased in as attendees arrive. This facility will
be open between the hours of 10:00 and 02:00 over the course of the event. Outside
of these hours, an out-of-hours number will be available to call an on-call first aider.
This will be advertised to all attendees and on-duty staff will be made aware.

This first aid point will be equipped with Basic Life Support (BLS) equipment including
an AED, oxygen therapy, and medical interventions. The first aid point will also operate as
a welfare point, equipped with bottled water, sun cream, camp beds and heaters.

A registered paramedic will be on site in possession of a full ALS kit, including advanced
interventions. This paramedic will be on-shift for 12 hours per day during the event. At
night, they will maintain an on-call presence using telephone contact.

HCPs (Doctors, Paramedics and Nurses) will be rotated through on-call cover during the
event. An on-call HCP will be available at all times to handle any adverse incidents.

Responder teams consisting of two first aid trained members of staff will patrol the event
during opening hours. They will patrol with a large BS 8599-1 first aid kit and will receive
backup of advanced BLS equipment as required.

Pre-cover for the event will be undertaken by on-site event staff who have undertaken first aid
at work qualifications with additional qualifications which include working from height, plant
and electrical safety.

\subsection{Access}

Access to the site is via road. The first aid point will be located near the
entrance to the site via access route A. In the event that an ambulance has
to enter or leave the site, the first aid team will liaise with the stewards to ensure
the area of road around the site entrance is cleared.

Non-urgent casualties who require hospital treatment but not an ambulance will
be taken to the receiving hospital via car, either by a car belonging to the
first-aid team, or a taxi. In both cases, the casualty will be escorted by a
member of the first-aid team, if requested to do so by the casualty or if the
first-aider deems it necessary.

The site itself is a mixture of fields and tracks. There are
trackways throughout the site, with extra fire lanes marked for emergency
vehicles. In the event an ambulance is required at another part of the site,
the first-aid team will liaise with the stewards to direct the vehicle and
provide clear access.

\subsection{Communications}

The First-aid team will work closely with the volunteer, info-desk and childcare
teams.

Volunteers and stewards will be made aware of the location of the first-aid
point and the direct phone number for the first aid team. A single number will 
call all first aiders on duty. Volunteers will also be able to use a radio to 
contact first aiders.

Volunteers and members of the general EMF team who respond to a first-aid
situation should advise the casualty to make their way to the first aid point
if they are able to do so. If not, the volunteer should contact the first aid
team immediately and stay with the casualty until members of the first aid team
arrive.

Between the hours of 02:00 and 10:00, the first-aid team will be available via the
first aid phone number. Volunteers responding to first aid situations at this
point should proceed as above, but call the first aid phone number
immediately.

Both the first aid point and the roaming team will be provided with a radio.
Radio protocols, such as channels, codewords and such will be circulated at the
initial briefing, on-site. Particular attention is given to the lost and found
child reporting (see Lost Children Policy).

\subsection{Hospitals}

The nearest hospitals with A\&E departments to Eastnor Castle Deer Park are:

\begin{itemize}
    \tightlist
\item Gloucestershire Royal Hospital, Gloucester: 13.2 miles
\item Worcestershire Royal Hospital, Worcester: 13.7 miles
\item The County Hospital, Hereford: 14.3 miles
\end{itemize}

\subsection{Air Ambulance}

A clear field will be kept for air ambulance landings in the Deer Park south of
the event, in an area which is believed to be an existing HEMS secondary landing
site.

\subsection{Equipment}
Equipment will be provided for each first aid volunteer. These will include a
first aid kit, a high-visibility tabard and a radio. Volunteers will be advised
to bring appropriate warm and waterproof clothing and footwear.

\subsection{Waste disposal}
Clinical waste will be disposed of in the correctly marked bags and
removed by the ambulance contractor. Sharps bins will be provided and
disposed of in the same fashion if used.

Standard waste will be disposed of using the facilities provided by the sanitation team.

\subsection{Qualifications}
Volunteer first aiders must have, as a minimum, a First Response Emergency Care
level 3 (FREC3) or First Person on Scene (FPOS) qualification.

A copy of the qualification will be checked by the First Aid Team Lead
before the event. A digital copy will be retained in a secure format in accordance
with the Data Protection Act.

\subsection{Safeguarding}

As EMF is not a first aid organisation, there is no legal requirement for
volunteers to be DBS checked under the Safeguarding Vulnerable Groups Act 2006
(as amended), with respect to either children (Schedule 4, 2(3B)(1)) or vulnerable
adults (Schedule 4, 7(3A)).

EMF will endeavour to obtain DBS checks for all first aid volunteers. Registered
health care professionals will be assumed to be adequately checked by virtue of
their professional registration.

In the event that a volunteer cannot provide a DBS check due to inadequate time
available, they will be paired with another person for all first aid duties.
This will only apply when a DBS check is clearly unable to be completed due to
lack of time available, and not when the volunteer has had adequate time to
facilitate the check.

Notwithstanding the above, first aiders will work in pairs whenever possible, and
only DBS-checked volunteers or healthcare professionals will be permitted to work
alone, for example when on call overnight.

Trained childcare staff are also available on site as part of the youth area.

\subsection{Insurance}
EMF will carry medical malpractice/indemnity insurance cover for all first aiders.

\subsection{Reporting}
All first aid administered will be recorded on an electronic or paper
patient report form, with any serious incidents being reported on an
additional RIDDOR form. Patient report forms will be kept securely by the First
Aid Team Lead, with a copy sent to the patient upon request. This
data will be kept in a secure format and accordance with the Data Protection
Act.

These members of the first aid team who are members of St John, Red Cross or another
organisation who have their own paperwork requirements may fill out their own paperwork
in addition to EMF's requirements.

\subsection{Local Authorities}
Both the local ambulance service control room and the local police service will
be informed, prior to festival, that an event is taking place where first aid
cover is being provided.

\subsubsection{Social Services}

Given the nature of the event, it is recommended that local social services be
made aware of EMF.

\subsection{Major Incidents}

In the event of a major incident occurring on site, all personnel, unless
actively treating a patient, are to report to the First Aid Post for
instructions. This includes these first aid volunteers who are currently on
shift. The first aid lead on duty should immediately report the situation to the
Duty Site Manager.

All EMF medical resources will be placed under the direction of the Ambulance
Service on request.

Any fatalities will be dealt with by the Police in conjunction with the Event
Management Team and the NHS Ambulance Service in line with the Police Plan for
this eventuality.
